%!TEX root = CV_GIULIO_DE_MATTIA.tex

% --- Dati personali ---
\name{Giulio Romano De Mattia}
\role{Compositore \& Live Electronics}
\contacts{Roma}{+39 342 3232 342}{giuliodemattia21@gmail.com}{https://github.com/DMGiulioRomano \\ https://soundcloud.com/giulio-romano-de-mattia}
\photo{img/fotoInfinitiPossibili} 

\makeheader
% --- Sommario ---
\summary{Compositore e live electronics specializzato in spazializzazione multicanale e regia del suono per musica contemporanea. Ho curato prime assolute e installazioni in contesti istituzionali (Festival ArteScienza a cura del CRM di Michelangelo Lupone, Conservatori S. Cecilia e A. Casella), con focus su feedback e interazione spazio--ambiente. Conduco laboratori di musica d'insieme con un attenzione particolare al timbro e alla musica elettronica nelle scuole Medie e nell'Infanzia. Produco tool audio open source in Csound, Max/MSP, Pure Data, JUCE e FAUST con workflow documentati su Github.}
% --- Competenze ---
\skills{
  \item Live electronics \& spazializzazione (Ambisonics, MS)
  \item Regia del suono per ensemble e acusmatico
  \item Progettazione catene elettroacustiche e feedback systems
  \item Sviluppo tool audio (Max/MSP, PD, JUCE)
  \item Misure acustiche e calibrazione multicanale
  \item Documentazione tecnica e release open source
}

% --- Esperienze lavorative ---
\cvsection{Esperienze Professionali}{
\experience{06/2023 -- oggi}{ArteScienza / CRM, Roma}{Assistente tecnico-musicale}{
  \item Regia del suono e live electronics per prime assolute (16--24ch)
  \item Ottimizzazione acustica sala/installazioni
  \item Coordinamento palco e tecnica con Goethe-Institut
}

\experience{10/2024 -- oggi}{Laboratorio bambini 3--5 anni, Roma}{Conduttore laboratorio}{
  \item Ideazione e conduzione di laboratorio timbrico (8--12 partecipanti)
  \item Micro-moduli di ascolto attivo e documentazione audio
}

\experience{10--11/2023}{Easylight S.R.L.}{Fonico \& tecnico luci}{
  \item FOH per conferenze/spettacoli (platea 150--400)
  \item Inventariazione e controllo attrezzature, ripristino asset in 24h
  \item Supporto lighting (20--40 fixture)
}
}

% --- Istruzione ---
\cvsection{Formazione}{
\education{2023}{Conservatorio "S. Cecilia", Roma}{Diploma I livello, Musica Elettronica (108/110)}{Tesi: Rapporto compositore--interprete nella musica elettroacustica. Docenti: N. Bernardini, P. Citera, G. Silvi}
}

% --- Eventi artistici ---
\cvsection{Attività Artistiche}{
% Sintassi: \artisticevent{data}{luogo}{ente organizzatore}{titolo evento}{organico}{programma}

\artisticevent{07/2023}{Goethe-Institut, Roma}{ArteScienza/CRM}{AEDI: *A Memory Mistake*}{Acusmatico 16 canali}{Prima assoluta di composizione originale per sistema multicanale con Olofoni}

\artisticevent{09/2024}{LEAP, Roma}{CRM}{Jolt}{Campana tibetana + live electronics}{Trasformazioni in tempo reale, focus su risonanze profonde e spazializzazione}

\artisticevent{10/2024}{Conservatorio A. Casella, L'Aquila}{Conservatorio A. Casella}{Nono: *À Pierre. Dell'azzurro silenzio, inquietum*}{Flauto, clarinetto contrabbasso + live electronics}{Integrazione catena elettroacustica per ensemble da camera}
}

% --- Lingue ---
\section*{Lingue}
Italiano: madrelingua \\ Inglese: C1 \\ Tedesco: B1

% --- Consenso GDPR ---
\section*{Consenso}
Autorizzo il trattamento dei dati ai sensi del Reg. UE 2016/679.