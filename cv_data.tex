% File: cv_data.tex (generato automaticamente)

% --- Dati personali ---
\name{Giulio Romano De Mattia}
\role{Compositore e Live Electronics}
\contacts{Roma}{+39 342 3232 342}{giuliodemattia21@gmail.com}{https://github.com/DMGiulioRomano \\ https://soundcloud.com/giulio-romano-de-mattia}
\photo{img/fotoInfinitiPossibili}

\makeheader% --- Competenze ---
\skills{
  \item Live electronics e spazializzazione (Ambisonics, MS)
  \item Regia del suono per ensemble e acusmatico
  \item Progettazione catene elettroacustiche e feedback systems
  \item Sviluppo tool audio (Max/MSP, PD, JUCE)
  \item Misure acustiche e calibrazione multicanale
  \item Documentazione tecnica e release open source
}

% --- Esperienze ---
\experience{06/2023 -- oggi}{ArteScienza / CRM, Roma}{Assistente tecnico-musicale}{
  \item Regia del suono e live electronics per prime assolute (16-24ch)
  \item Ottimizzazione acustica sala/installazioni
  \item Coordinamento palco e tecnica con Goethe-Institut
}

\experience{10/2024 -- oggi}{Laboratorio bambini 3-5 anni}{Conduttore laboratorio}{
  \item Ideazione e conduzione di laboratorio timbrico (8-12 partecipanti)
  \item Micro-moduli di ascolto attivo e documentazione audio
}

\experience{10-11/2023}{Easylight S.R.L.}{Fonico e tecnico luci}{
  \item FOH per conferenze/spettacoli (platea 150-400)
  \item Inventariazione e controllo attrezzature, ripristino asset in 24h
  \item Supporto lighting (20-40 fixture)
}

% --- Istruzione ---
\education{2023}{Conservatorio "S. Cecilia", Roma}{Diploma I livello, Musica Elettronica (108/110)}{Tesi: Rapporto compositore-interprete nella musica elettroacustica. Docenti: N. Bernardini, P. Citera, G. Silvi}

% --- Attività selezionate ---
\artisticevent{11/07/2023}{Goethe-Institut, Roma}{ArteScienza/CRM}{AEDI: A Memory Mistake}{Acusmatico}{Setup multicanale con Olofoni di Michelangelo Lupone}

\artisticevent{01/03/2024}{Conservatorio S. Cecilia, Roma}{Conservatorio S. Cecilia}{AEDI: Transience}{Clarinetto Basso (Alice Cortegiani), Tape}{Coordinamento di 20 persone, curatela completa dell'evento}

\artisticevent{13/07/2024}{Goethe-Institut, Roma}{ArteScienza/CRM}{Post-Prae-Ludium per Donau (Luigi Nono, 1987)}{Euphonium (Marina Boselli), Live Electronics (con Francesco Ferracuti)}{Repertorio - Live Electronics}

\artisticevent{18/09/2024}{Goethe-Institut, Roma}{ArteScienza/CRM, LEAP}{Jolt}{Campana tibetana (Marco Di Gasbarro), Live Electronics}{Curato da Francesco Vitucci e LEAP}

\artisticevent{19/09/2024}{Liceo Quirino Visconti, Roma}{CRM, Liceo Visconti}{Toccare L'invisibile}{}{Progetto tra Liceo Visconti e Centro Ricerche Musicali}

\artisticevent{10/10/2024}{Conservatorio A. Casella, L'Aquila}{Conservatorio A. Casella}{À Pierre. Dell'azzurro silenzio, inquietum (Luigi Nono)}{Flauto Contrabbasso (Giuseppe Silvi), Clarinetto Contrabbasso (Alice Cortegiani), Live Electronics}{Integrazione catena elettroacustica}

\artisticevent{11/10/2024}{Conservatorio A. Casella, L'Aquila}{Conservatorio A. Casella}{AEDI: Transience}{Clarinetto Contrabbasso (Alice Cortegiani), Tape}{Compositore, regia del suono}

\artisticevent{11/10/2024}{Conservatorio A. Casella, L'Aquila}{Conservatorio A. Casella}{Saxony (James Tenney)}{Clarinetto Contrabbasso, Live Electronics}{Live Electronics, regia del suono}

\artisticevent{19/12/2024}{La Pelanda, Ex Mattatoio, Roma}{Nuova Consonanza}{Umano Post Umano}{}{Opera di Agostino Di Scipio - Gestione fusione suono acustico ed elettronico per teatro musicale}

\artisticevent{20/01/2025}{Auditorium Parco della Musica Ennio Morricone, Roma}{PMCE}{Viaggio in compagnia di Gavin Bryars}{Con Raiz, Conservatori del Territorio}{Studenti di Composizione (Maestro Vittorio Montalti) e Musica Elettronica (Maestro Agostino Di Scipio)}

% --- Lingue ---
\section*{Lingue}
Italiano: madrelingua \\\\ Inglese: C1 \\\\ Tedesco: B1

% --- Consenso GDPR ---
\section*{Consenso}
Autorizzo il trattamento dei dati ai sensi del Reg. UE 2016/679.
